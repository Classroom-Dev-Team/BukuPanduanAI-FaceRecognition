\section{Face Detection}
\emph{Face Detection} atau pengenalan wajah merupakan sebuah teknologi untuk menangkap wajah seseorang pada kamera yang menjadi tahap awal dalam sistem pengenalan wajah (\emph{Face Recognition}) 
yang digunakan dalam identifikasi biometrik. Deteksi wajah juga dapat digunakan untuk pencarian atau pengindeksan data wajah dari citra atau video yang berisi wajah dengan berbagai ukuran, posisi, dan latar belakang.
\footnote{NUGROHO, Setyo, Drs. Agus Hardjoko, MSc.,PhD. \emph{Sistem pendeteksi wajah manusia pada citra digital}, Universitas Gajah mada, diakses dari http://etd.repository.ugm.ac.id/penelitian/detail/23416}\\

Pembuatan pendeteksi wajah ini dapat dibuat menggunakan openCV yang  merupakan aplikasi perangkat lunak untuk pengolahan citra dinamis secara \emph{real-time}, selain itu openCV juga banyak mendukung 
bahasa pemrograman diantaranya C++, C, python, dan java. Pada pembahasan kali ini, penjelasan mengenai proses pembuatan deteksi wajah akan menggunakan openCV dengan bahasa pemrograman python. 
Proses deteksi objek maupun wajah dapat menggunakan metode algoritma Haar Cascade Classifier.\\

Algoritma Haar Cascade Classifier merupakan salah satu algoritma yang digunakan untuk mendeteksi sebuah wajah dengan cepat dan \emph{real-time} sebuah benda termasuk wajah manusia. 
Metode ini menggunakan haar-like features dimana perlu dilakukan training terlebih dahulu untuk mendapatkan suatu pohon keputusan dengan nama cascade claasifier sebagai penentu 
apakah ada obyek atau tidak dalam frame yang diproses, dengan mengelompokka fitur-fitur pada gambaran wajah berdasarkan sisi yang terang dan sisi yang gelap. Adanya fitur Haar
ditentukan dengan cara mengurangi rata-rata piksel pada daerah gelap dari rata-rata piksel pada daerah terang\footnote{Suhepy Abidin.\emph{Deteksi Wajah Menggunakan Metode Haar 
Cascade Classifier Berbasis Webcam Pada Matlab}.Jurusan Teknik Elektro, Politeknik Negeri Ujung Pandang}

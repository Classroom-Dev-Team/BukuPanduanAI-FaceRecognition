\section{Artificial Intelligence Pada Pengenalan Wajah}
 Dilansir dari Stanford Computer science, Artificial Intelligence(AI) atau kecerdasan buatan adalah ilmu dan rekayasa pembuatan mesin cerdas, melibatkan mekanisme untuk menjalankan
 suatu tugas menggunakan komputer.  Sehingga artificial intelligence merupakan sebuah teknologi yang memungkinkan sistem komputer, perangkat lunak, program dan robot untuk “berpikir” 
 secara cerdas layaknya manusia. Kecerdasan buatan suatu mesin dibuat oleh manusia melalui algoritma pemrograman yang kompleks.\footnote{Mustofa, Zaenal. \emph{Artificial Intelligence (AI): 
 Pengertian, Perkembangan, Cara Kerja, Dan Dampaknya}.Universitas STEKOM} \\
\\Secara garis besar, AI dapat melakukan salah satu dari keempat faktor berikut:
\begin{enumerate}[a.]
    \item \emph{Acting Humanly} , sistem bertindak layaknya manusia.
    \item \emph{Thinking Humanly} , sistem dapat berpikir seperti manusia.
    \item \emph{Think Rationally} , sistem dapat berpikir secara rasional.
    \item \emph{Act Rationally} , sistem mampu bertindak secara rasional.
\end{enumerate}

Pengenalan dan identifikasi wajah merupakan contoh sistem penerapan konsep Artificial Intelligence menggunakan biometrik wajah yang terus berkembang pada bidang \emph{computer vision}. 
Kecerdasan buatan ini digunakan secara \emph{real-time} untuk menangkap dan mengenali wajah seseorang pada kamera.

Computer Vision adalah bagaimana komputer/mesin dapat melihat, teknik computer vision mampu memvisualisasikan data menganalisaberupa gambar image atau dalam bentuk vidio. Tujuan utama dari 
Computer Vision adalah agar komputer atau mesin dapat meniru kemampuan perseptual mata manusia dan otak, atau bahkan dapat mengunggulinya untuk tujuan tertentu.
\footnote{Wibowo, Ari.\emph{Implementasi Teknik Computer Vision}.Universitas Widyatama}
